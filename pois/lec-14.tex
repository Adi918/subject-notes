\section{Generalized Secret Sharing}

(a beginning towards a major unresolved
problem/issue in Crypto/Algo)

\section{Review of last class}
- Shamir's secret sharing, and using it for secure
multi-party communication.


\section{Kannan philosophy}

We will take generalized secret sharing and generalise it. This will crop
up to have ramifications on crypto and algo.

\section{Generalization}

We have a secret $S$ that we split into $n$ shares $S_1\cdots S_n$.

In Shamir's secret sharing, $\leq t$ learns nothing, $\geq t + 1$ learns
full information.

Implicit assumption: Network is homogeneous in trust. Adversary will attack
one part of the network just as likely as any other part of the network.


However, there can be situations where we believe that some subset of the
network is relatively more trustworthy than other sections of the network.

In this case, we want to get $n$ shares, such that $<= t_1$ in the first
$\frac{n}{2}$, $<= t_2$ in the next $\frac{n}{2}$ does not get the secret.
Other combinations can learn.

Example: let $n = 4$, $t = 2$, this means all subsets of size $>= 2 + 1$
can access the secret.

Example': we want to make a statement such as: This basis (aka "access structure")
and all supersets of the basis should be allowed to access the secret:
Eg, we can give a basis 
{{S1, S2}, {S1, S3, S4}}, and we want a secret sharing method that will let
us share secrets among these sets and their supersets.


Note that this *is a generalization* of the original. A $t$ access threshold is
this basis scheme, where the basis is all subsets of size $t + 1$.


\section{Alternate view of access structure as boolean functions}
We said that an access structure is *monotone* over over subsets of ${1..n}$ 
(MONOTONE: If a subset $S$ is in the access structure, all supersets of $S$
are in the access structure)


We can look at the access structure as $f: {0, 1}^n \rightarrow {0, 1}$. (There
is clearly a bijection between subsets and ${0, 1}^n$, so represent subset as
${0, 1}^n$. We define  $f(bitencode(subset)) = 1$ if subset is in access structure, $0$ otherwise.


Now, we need $f$ to be a monotone boolean function. that is,
$f(bitencode(S)) = 1 \implies forall S \subset S', f(bitencode(S')) = 1$ (if $S$ can access the secret,
all supersets of $S$ can access the secret).

So now, we have constructed a boolean function to represent our access structure.
We will now invoke complexity theory.

\subsection{Hardness}
We can construct an access structure which will be lower bounded in exponential
size of the secret?

Number of subsets of powerset of set $n$ is $2^(2^n)$.
 
The monotonicity does not reduce this by much (Sid: Proof?)

For every access structure, we will have a secret sharing scheme. So, we will
have $O(2^2^n)$ secret sharing scheme. So, the number of bits for some
secret sharing scheme will be $O(log(2^2^n)) = O(2^n)$.


So, we will  have a secret sharing scheme that is exponential. Note that
the length is the number of instructions in the scheme  (both the message itself
and the instructions for the secret sharing) will be exponential. However, the
*share* could be small.

So, we have a computation versus communication trade-off that we need to explore.
That is, we can trade-off the sizes of the computation (instructions length)
and the size of communication (that is, the share length).



\subsubsection{Unresolved Problems (Open problems)}
\begin{itemize}
    \item Does there exist an access structure on $n$ shares 
        such that *every* secret sharing scheme for it has super-polynomial
        length? (that is, is there an access structure that does not allow
        for efficient encoding in terms of share length).

    \item Suppose we convert an access structure to a monotone boolean function.
        Suppose we only care about those functions that are in $P$ 
        (that is, polynomial circuit depth).

        Do all efficiently computable (in $P$) access structures have
        efficient secret sharing schemes?

        Given an access structure that is *efficiently computable*
        (in $P$), can we construct a secret sharing scheme that is in P?


        Relationship to algorithms:

        We have a notion of "input size" in algorithms.
        If we have a distributed algorithm, assume it is parametrised by
        number of nodes $n$, and say $<= t$ nodes that are allowed to be faulty.


        Assume we had captured the fault tolerance in terms of our
        boolean function, $f: {0, 1}^n -> {0, 1}$. When
        $f(S) = 0$, it is faulty, and failure here need to be tolerated.
        When $f(S) = 1$, it is not faulty, and failure here need not be tolerate.

        Note that these are equivalent to the old definition. Sets with
        $f(S) = 1$ are the "critical nodes", which need to be present. We can
        have $f(S) = 1 forall |S| >= t + 1$, and this gives us back our
        old definition based on number of nodes that are faulty.

        So, distributed systems can also say, we wish to tolerate some
        monotone $f$ in general, so we have generalized distributed system
        tolerance to a general "critical nodes" notion.

        So now, the size of our description of our distributed systems
        algorithm depends on the **size of $f$** (Since $f$ is now a
        parameter to our distriuted system algorithm)

        Before, our distributed system algorithm was $dsalgo(n, t)$, Now it is
        $dsalgo'(n, f)$, so we need to give f as a parameter.


        Consider the old problem with tolerance described in terms of the
        number $t$. Encoded as a function, it is: $f(S) = 1 if |S| > t, 0 when |S| <= t$.
        If we have to *describe* f, then if we decide to describe it in terms
        of the basis, we will have $nC(t + 1)$ elements in the basis. 
        What used to be a **number**(t) is now a set of size $nC(t + 1)$, since
        we generalized $t$ to $f$.

        
        If I choose to view fault tolerance as an access structure problem
        always, then the basis size is in itself $nC(t + 1)$. So, my
        input size is $nC(t + 1)$. So, any ridiculous crap of $nC(t + 1)$ 
        will be "polynomial". 

        However, the old encoding will consider a polynomial in $nC(t + 1)$
        as exponential.

        So, our choice of encoding is skewing our notion of what is
        "small".

        So, we need some reasonable way to define "size of access structure",
        that does not allow blowup like this.


        Why can't we argue that the input size is size of the function $f$?
        Size of $f$ has two notions: one as the length of the description of $f$,
        and the other as the *time taken for $f$ to execute*. That is, 
        I can have a small description that takes exponential time, or I can
        have an exponential description that takes constant time.
        (list of tuples verus encoding the smallest program)

        This leads us to: what is the size of a program? If two programs
        have the same length, but one is faster, then we would like to 
        consider the one that is faster (maybe?)

        Why should runtime play a role? Kannan argument: of what use is a program
        that will not give us output for billions of years. This is as good as
        not giving us a program at all.

        So, we would like to optimise on both length and running time.


        This fucks us over when the program that we choose is used to
        describe something about our runtime environment, like the access
        structure as given above.


        This naturally leads us to the question, given that the access
        strucrure is efficiently computable, can we have an efficient
        secret sharing scheme for that?


    \item Assume that it is *impossible* to have efficient secret sharing
        schemes for efficient access structures, so we wish to crypto it
        and solve-the-impossibility.

        We do not even know if we can solve this assuming the whole crypto-model.
        That is, adversary is PPTM, negligible prob. of error, one way
        functions exist, do all efficient access structures have efficient
        secret sharing schemes? Unknown.

    \item Supposedly, similar problems crop in other areas. To quote kannan,
        "messy waters", "murky", etc.


        Coding theory:

        We have a noisy channel, and we have to model noise.
        Noise is modelled as: if I send $n$ bits/symbols, $<= t$ symbols
        are in error. Can we give an error correction code that can 
        tolerate such noise?

        Again, do the same thing, replace $<= t$ with an access structure, as
        we did in distributed systems.  We land in the same problem of
        describing input size.


        Why would we need access structures for ECC ever? For example, consider 
        two channels between $S --> R$. One channel has 1/5 chance of error,
        other channel has 1/4 chance of error. Say I send the first n/2 bits
        over the first channel and the next n/2 bits over the next channel.

        This can be modeled using an access structure over the full n bit
        string. So, knowing details about where this "toggling" will happen
        should let us design better ECC schemes.

        For example, if our channel sent first n/2 bits correctly and next
        n/2 bits fully wrong, we have an ECC: only send data in the first n/2
        bits. This is *not* the same as stating that "50% of data is corrupted"
        In this case, there is no ECC that can solve this.

        So, having data regarding where the toggles happen is *useful information*
        to have. So, the access function is a *useful abstraction* in ECC, it
        is not frivolous.

        Thus, we are not allowed to say "access structures are useless in ECC".
        So now, our noise is a program / monotone boolean function.  This now
        leads us to the thorny problem of defining size of monotone
        boolean function.


        For the past half-century, ECC has only worked on the threshold
        kind of $f$. We have not worked on ECC of the general access
        function kind.

        Rounding back, we need to define input size of $f$, which is the
        noise in the channel.
        
        If $f$ is very short in length but it takes exponential time to run, then we cannot
        "observe" the error since it takes exp time to run.

        Suppose we get a noisy channel in real life whose $f$ is not in $P$. Then the
        real world Channel is computing $f$ which is in NP.
        
        So, we can sample the channel (which is computing $f$) to solve NP
        problems. But such channels should not exist, since nature
        cannot solve NP problems in P time. So, we need
        $f$ which is in $P$.

        Q) does there exist efficient ECC for all realistic channel f \in P?

        Cute Scott Aaronson quote: "If P != NP were a question in physics, then
        it would have been a law by now (since we have *never* observed nature
        solve NP problems in P time".

        
\end{itemize}


Physics is ridiculous, since we expect reality == math prediction.

We can weaken it, by saying people should be able to distinguish reality 
and math prediction in polynomial time!
