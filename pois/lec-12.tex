\section{How to query a DB without revealing data / Oblivious Transfer}

1. We shoul not reveal the query
2. Database reveals nothing except for the query answer to the query


\section{Simplification of the problem}
Consider an array of $n$ bits, $Arr = b_0 b_1 \cdots b_n$. The query is an index query $i$.
The task is that the querier $B$ should get to know $b_i$, should not get to know
$b_j, j \neq i$. $A$ should not *know $i$*!

Intuitively, it seems like ``information deadlock'' should take place. Since neither
party knows what to do, there is some sort of deadlock. Hence, impossibility. (Kannan
is happy here, since now we can information theory this, as he puts it).

Supposedly, one-way-functions will work.

$x \rightarrow f(x) is easy$.
$f(x) \rightarrow x is hard$. If we know trapdoor information,
$f(x) \rightarrow x$ is easy.

\section{Construction of the scheme}

$f: [1, n] -> [1, n]$ is a trapdoor one-way *permutation* (unique decryption).

\begin{itemize}
\item 1. $B$ chooses $n$ bits at random - $r_1 r_2 \cdots r_n$.
\item 2. $B$ applies $f$ only at $r_i$ to obtain $Z = r_1 r_2 \cdots r_{i -1} f(r_i) r_{i + 1} \cdots r_n$.
  $f$ may not be applicable to single bits. In this case, we can use the hadcore predicate and $XOR$ $r_i$ with the
  hardcore predicate.
\item 3. $B$ sends $Z$ to $A$.
\item 4. $A$ decrypts (find inverse) of $Z$ and obtains $Y = f^{-1}(r_1), f^{-1}(r_2), \cdots r_i, f^{-1}(r_{i + 1}), f^{-1}(r_n)$
\item 5. Let array be $Arr = b_0 b_1 \cdots b_n$.
\item 6. Perform $Arr XOR Y = b_0 XOR f^{-1}r_1, \cdots , b_{i} XOR r_i, \cdots b_n XOR f^{-1}(r_n)$.
\item 7. A sends $Arr XOR Y$ to $B$.
\item 8. $B$ obtain $b_i = (Arr XOR Y)[i] XOR r_i = (b_{i} XOR r_i) XOR r_i = b_i$.
\end{itemize}

Note that $B$ cannot see $b_j$ where $j \neq i$, since in some sense, we have ``encoded'' the $b_j$ with  $f^{-1}$.

$A$ cannot know which index is the correct index, since there is no ``marker'' for the correct index.





\section{Solving the universal problem}

$A$ has input $X$. $B$ has input $Y$. we wish to compute $Comp(X, Y)$. Either both of
them want $Comp(x, y)$, or one of them want $Comp(x, y)$.

$A$ and $B$ are unwilling to reveal their information to each other.

So, how does one solve this? Generalize our specialized construction.

Andrew Yao was awarded the Turing award in 2000 for posing the general problem and solving it.


\section{Yao's millionaire problem}

There are two millionares (Kannan quip: let me make them billionaires, because inflation). They wish to
find out who is richer, without revealing their bank balance to each other. Can we solve this?


\subsection{Weird kannan style generalization}

Given two machines $A$ and $B$, can we construct a virtual machine $S$, such that $A, B$ do not know what
$S$ is computing, but they virtually simulate $S$?

That is, $Mem(s) = Mem(a) XOR Mem(b)$

Can a cluster of insecure machines simulate a secure machine? (Neat!)

Sid question: Can we not construct FHE by keeping some data on the client as well? It could be redundant data,
but it would still be part of the algorithm? I guess this does not really give you FHE, because the full data is not
owned by one party.





\subsubsubsection{Kannan tangent - Teaching ethics instead of teaching crypto}
It is better to teach ethics and forego the area of crypto, rather than teach crypto and allow people
to forget ethics.

I'm not keen on crypto solving dishonesty problems.

\begin{itemize}
\item 1. There will be dishonest people in the world
\item 2. Software will have bugs for dishonest people to exploit.
\end{itemize}

Outside of Earth, it has already caught on.

The first major implementation of our solution was performed by satellites (?) (what in the hell, TIL).
They don't want to collide in mid-space, but they **do not want to reveal where they are**.

As satellite traffic increased, there was a genuine chance that collision would take place.
They ran this protocol between satellites so they can prevent collisions.


\subsubsection{Finally, the solution}

We wish to perform an instruction $z <- x + y$ on $S$, where $x, y$ are also stored in $S$.

\subsubsubsection{Constructing XOR}
$x$ is stored in $S$ means that $x_a$ is in $A$, $x_b$ is in $B$$, $x = x_A XOR x_b$.
(note that in our scheme, $x_a$ and $x_b$ are stored *at the same memory loc at $A$ and $B$.
That is $A_{ram}[i] = x_a, B_{ram}[i] = x_b$.


Party $A$ performs:
$z_a= x_a XOR y_a$

Party $B$ performs:
$z_b = x_b XOR y_b$



\subsubsubsection{Construct AND}
We know what the output should look like:
$z_a XOR z_b = (x_a XOR x_b) /\ (y_a XOR y_b)$


Code for A:
$z_a <- random {0, 1}$. 
A creates an array of length $4$, which stores values of $z_b$ corresponding to values of $x_b, y_b$.
(Look at $z_A XOR z_b$, and see what the value should look like). So, consider all 4 cases, corresponding
to the indeces of $A_i$.

0. If x_b = 0, y_b = 0,
z_b = (x_a /\ y_a) XOR z_a. = Arr_0


1. If x_b = 0, y_b = 1,
z_b = (x_a /\ (NOT y_a)) XOR z_a = Arr_1.


2. If x_b = 1, y_b = 0,
z_b = ((NOT x_a) /\ y_a)) XOR z_a = Arr_2.


3. If x_b = 1, y_b = 1,
z_b = ((NOT x_a) /\ (NOT y_a)) XOR z_A = Arr_3.


Code for B:
Run oblivious transfer with $n = 4$. $A$ has a linear database of size 4. $B$ has the index.
Hence, $B$ gets $z_b$.


\subsubsubsection{Note: why pick $z_a$ randomly?}
$z_a$ acts as one-time-pad in the array table construction. This obscures
what $B$ can see about $x_a$, $y_a$.

\subsubsubsection{Exploiting multiple machines - Byzantine situations}

With many machines, we can XOR the data between many machines. This gives us much higher security.
However, we lose out on fault-tolerance. People have explored this fully, and we can ask for arbitrary fault tolerance and security,
and we can then recieve a protol to be used for that setting.

We can have at most $n/3$ parties that were colluding and disrupting among a cluster of $n$ machines, and still have
fault tolerance and security.

