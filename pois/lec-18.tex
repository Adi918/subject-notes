\documentclass[11pt]{article}
%\documentclass[10pt]{llncs}
%\usepackage{llncsdoc}
\usepackage{amsmath}
\usepackage{graphicx}
\usepackage{makeidx}
\usepackage{algpseudocode}
\usepackage{algorithm}
\evensidemargin=0.20in
\oddsidemargin=0.20in
\topmargin=0.2in
%\headheight=0.0in
%\headsep=0.0in
%\setlength{\parskip}{0mm}     
%\setlength{\parindent}{4mm}
\setlength{\textwidth}{6.4in}
\setlength{\textheight}{8.5in}
%\leftmargin -2in
%\setlength{\rightmargin}{-2in}
%\usepackage{epsf}
%\usepackage{url}
\usepackage{epsfig}
\usepackage{tabularx}
\usepackage{latexsym}
\newtheorem{lemma}{Lemma}
\newtheorem{observation}{Observation}
\newtheorem{proof}{Proof}
\newcommand\ddfrac[2]{\frac{\displaystyle #1}{\displaystyle #2}}

\def\qed{$\Box$}
\def\proof{\textit{Proof. }}
\newtheorem{corollary}{Corollary}
\newtheorem{theorem}{Theorem}

\begin{document}
\section{Cryptocurrencies}

\section{Why are they impossible? (If they were possible, would not be part of the course!)}

Consider some cash. Once we spend it, we don't \textit{own it anymore}.
Copying physical currency is difficult, expensive, and innacurate.
Hence, double spending is difficult.

Copying digital media is possible. Hence, we can construct a central authority to keep
track of money. However, this causes us to lose out on decentralisation, and also lose out on
annonymity.


In a P2P system however, since agreement is ``eventual'', we can construct scenarios of
double spending. This is a famous problem, apparently. 


\section{Idea behind cryptocurrencies}

All transactions are public. However, how does this solve double spending?

\section{Solving the problem: Blockchains}

1. Transactions are in chains (time series). A change at $t=i$ needs to make a change
for all $i < t0 \leq \text{len(chain)}$. That is, a change in the middle must cascade
till the end of the chain.

2. Chain must grow with new transactions.

Note that this is already making double spending difficult. If someone wants to mess with an older
transaction, he has to be faster than the rest of the world.

3. Takes a 'lot of work' to make a change.

4. Longest chain is the honest chain.

This makes double spending somewhat impossible: Changing the chain is now very hard. Since the fastest growing
chain represents ``truth'', an adversary will need to change the *longest* chain, as the longest chain grows.


Notice that it makes sense for the longest chain to be the honest chain, since in general, the longest chain is the
one that most people in the distributed system agree on (or you have access to more computational power than everyone else *combined*).


The proof-of-work is critical. We need proof that someone spent a lot of time solving a problem. So, we use
crypto puzzles for this. (verification is easy, solving is difficult. So, some NP problem).

People who add to the ledger are termed ``miners''.
A miner can add an arbitrary transaction to the blockhain, by being allowed to add a transaction that ``creates money''.
That is, he can add a transaction that says ``add 5 BTC to my account''. The block will need to contain proof-of-work,
as well as a link to the previous block, and a chain of transactions.


The current proof-of-work time is ~10 minutes for the entire network.


At any stage, the network will probably fork, for, say, 1 block. However, the probability of the network agreeing on a longest chain
is very high.

\section{How do blockchains solve our problems?}

When $A$ sends some money to $B$, $B$ waits for the transaction to appear in some longest chain. Once the transaction appears,
$B$ will render its services to $A$.

$A$ cannot double spend, since everyone knows that $A$ has sent money to $B$.

To invalidate the transaction $A \rightarrow B$, then $A$ needs to fork the chain before the transaction and establish a new
longesst chain. This is impossible since $A$ will be then competing against everyone else.


\section{Details of blockchain}



Notice that there is still a single point of failure: the CA that maintains our digital signatures.

\section{Possible solution that would be environmentally friendly}

We have a way to build secure computers on top of insecure computers. IT will probably be a hybrid of proof-of-stake
and proof-of-work.


There will be many miners. All miners will not be wasting effort. They will be involved in simulating one machine.
This abstract machine will choose which the next miners will be. Since we pick the next miners that will be a
subset of the previous miners, we won't waste work. However, the choice will be done in a way we that no one can
influence who the next person is.

Envionmental friendliness is not a lasting problem.

\end{document}
