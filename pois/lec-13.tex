\section{Secure multipart communication}

\subsection{Synchrony}
Existence of rounds. Send messages per round. Messages are
recieved by all per round.

Or, equivalently, there exists a global clock.

\subsection{Problem statement}

A, B, C have x_a, x_b, x_c inputs.

We wish to compute f(x_a, x_b, x_c) without revealing x_a, x_b, x_c.

For this, we **do not** need trapdoor one-way permutations!
Somehow, the existence of 3 people allows us to sidestep the
requirement of trapdoor one-way permutations.

We can simulate a trusted virtual server that is not under the control of
A, B, C, that can compute f(x_1, x_2, x_3) without revealing.

Adversary can only eavesdrop on **one of three parties** at any given time.
We do not know which party adversary is spying. Adversary has access to NP-oracle.

\subsubsection{Kannan philosophy}
The adversary is *omnipotent* (can solve NP complete problems in P). BUT, he
is not *omnipresent* (can only eavesdrop on one of A, B, C at a time).


\subsection{Machinery: Key management}

Can the key be stored in a network of $n$ memory spaces (called n ``shares'' of the key)
such that upto $t$ shares reveals nothing about the secret. all $t + 1$ or more shares
reveals the secret.

\subsubsection{Kannan philosophy: Rate of papers being published?}
Supposedly, when he had last seen this area and surveyed it a couple decades back, there were ~10,000 papers.
However, for some reason, we are unable to actually *see* this in our research life.

That is, per year, the course does not change by so much. why is it that the rate of increase of knowledge is
uncorrelated with the rate of papers being published?

He argues that most papers are trash, indirectly. ``We have learnt something about the art of generating
problems and solutions which is useful for training our mind, but not a contribution to the world.''

Maybe by the end of today we can see why that happens? (what? Does the crypto scheme shed some light on this?)

\subsubsection{Proof: Shamir's Secret Sharing scheme}
Consider FF, finite field, characteristic p. He picks $Z/pZ$. (Note: I will continue to use FF for finite field).

Pick a polynomial of degree $t$, with $t + 1$ coefficients.

$P(x) = \Sum_{i=0}^t a_i x^i$. $a_0 = S$. $a_i, \text{where} i > 0 = random element from F (Z_p in our case) $ .

The secret is $a_0 = S$.

The $i$th share is the polynomial evaluated at $i$.
$ Share_i = P(i), i = 1, 2, ... n$


Clearly, $t + 1$ or more shares reveals the secret, since we will have $t + 1$ points of a $t$ degree polynomial.

However, given only $t$ polynomials, we do not know anything about the constant term. We can look at this as a
generalization of a one-time-pad for $n$ terms.



\subsubsubsection{Consider an example for t = 2}

\begin{align}
P(0) &= S \\
P(1) &= S + a1 + a2 \\
P(2) &= S + 2a1 + 2a2
\end{align}

Having $P(1)$, $P(2)$ is not enough to reconstruct $P(0)$, the secret.


\subsubsubsection{Vandermonde matrix}
This construction can be seen as a vandermondle matrix:

$Share = A \cdot v$, where:


A = [1 1 1  1 1\\
  1 2 2^2  2^3\\
  1 3 3^2 3^3 3^3 \\
  1 4 4^2..]

$v = [S a_1 a_2 \cdots a_t]^T$.
$A_{ij} = i^j $. A has full rank, A is invertible.

\subsubsubsection{Hardness}
We know that $P[S=s] = \frac{1}{|F|}$ (since we pick each $S$ randomly. We use finiteness of F here).

Compute P{S=s | S_1=s_1, S_2=s_2} =? \frac{1}{|FF|}. If it is equal to \frac{1}{|FF|}, then it's a one-time-pad.

We take this on faith, and say that this will work. Lookup proof.


\section{Use machinery to solve secure multiparty communication}
In this case, we use the key management scheme to share our memory data.
We split our memory into $n$ shares, and give the $n$ shares to different parties.

We need $|FF| > n$, so that we can have a large field.

Now, we've built *secure memory*. We have yet to show that we can perform operations over this thing.

\subsubsection{Suppose $n = 3, t = 1$ (that is, any one should not get the secret, out of 3 machines)}


In this case, we have a 1-degree polynomial since $t = 1$. So, $P(x) = rx + s$.
(r = random, s = secret).

Hence, $S_a = r + s$, $S_b = 2r + s$, $S_c = 3r + s$.

We will construct an instruction set consisting of $add$, $multiply$ *over our FF* (kannan will be doing it over $Z_p$).

We will need $z <- x + y (in FF)$, $z <- x * y (in FF)$.


\subsubsection{Constructing $+$}

$x$ is shared according to $P_x(t) = r.t + x$,
$y$ is shared according to $P_y(t) = r'.t + y$.

A has x_a, y_a. B has x_b, y_b. C has x_c, y_c.
A should have z_a, B has z_b, C has z_c.

A computes: z_a <- x_a + y_a.
B computes: z_b <- x_b + y_b.
C computes: z_c <- x_c + y_c.

Note that:

P_x(1) = x_a.
P_y(1) = y_a.

P_x(2) = x_b.
P_y(2) = y_b.


P_x(3) = x_c.
P_y(3) = y_c.


We denote $P_{x + y} =(defined as)= P_x + P_y $.


x_a + y_a = P_x(1) + P_y(1) = (P_x + P_y)(1) = P_{x + y}(1)
x_b + y_b = P_x(2) + P_y(2) = (P_x + P_y)(2) = P_{x + y}(2)
x_c + y_c = P_x(3) + P_y(3) = (P_x + P_y)(3) = P_{x + y}(3)

Note that we have now computed a new secret sharing polynomial,
$P_{x + y}$. $z = P_x(0) + P_y(0) = x + y$ So, $z$ is encrypted with $P_{x + y}$.

\subsubsection{Constructing $*$ / Rabin Matching}
We denote $P_{x * y} =(defined as)= P_x * P_y $.
Note that we have now computed a new secret sharing polynomial,
$P_{x * y}$. $z = P_x(0) * P_y(0) = x * y$


$z <- x * y$
z_a <- x_a * y_a.
z_b <- x_b * y_b.
z_c <- x_c * y_c.

Polynomials are also homomorphic according to $*$
Hence, $z_a <- P_x * P_y = P_{x * y}$


However, this is a 2 degree polynomial! So, we now do not have P_{x * y}.

There is another problem: Shannon secret sharing gives us secrecy if the polynomial is *random*.
In this case, the polynomial is not random, since it is reducible (P_{x * y} = P_x * P_y). Reducible
polynomials are a subset of all polynomials (Sid question: what's the size of the subset?).


Now, can we somehow ``linearize'' P_{x * y} such that the  constant term remains the same,
and it is uniformly chosen across polynomials of degree t? (chosen over: $FF[x] / x^{t+1}$).


NOTE: I do not understand this final part properly!


We construct $z_a' = x_a * y_a$, $z_b' = x_b * y_B$, $z_c' = x_c * y_c$.
Then, we *share* z_a' as z_{aa}', z_{ab}', z_{ac}'. Repeat with z_{b}', z_{c}'.

They lie on a polynomial whose constant term is $z$.So,
\lambda_a z_a' + \lambda_b z_b' + \lambda_c z_c' = z.

We know that $z_a'$ is a linear combination of $z_{aa}'$, $z_{ab}'$, $z_{ac'}$
so, $z_a' = \lambda_a' z_{aa}' + \lambda_b' z_{ab}' + \lambda_c' z_{ac}'$.
Similarly for $z_b'$, $z_c'$.


So, $z$ will be a linear combination of all $z_{pq}'$ where $p, q \in {a, b, c}$

We can write this linear combination as:
z = $(\lambda_a \lambda_a' z_{aa}' +
     \lambda_b \lambda_a' z_{ba}' +
     \lambda_c \lambda_a' z_{ca}') +
     (same for b) +
     (same for c)$.

z = \lambda_a znew_a + \lambda_b znew_b + \lambda_c znew_c.

Note that znew_a, znew_b, znew_c are 1 degree.



--------
P_z(t) = r.t + z.

z_a = r + z
z_b = 2r + z
z_c = 3r + z

Find some linear combination of z_a, z_b, z_c such that we get z
z = t_a z_a + t_b z_b + t_c z_c
---------

OK, I don't know WTF happened. Read this tonight (21 feb 2018)



References for this:
1. BGW 1988 (in SoTC)
2. GRR 1998 (in ACM PoDC)

\subsubsection{Kannan philosphy}
Today is the first time we have brought in a different adversary. Usually, we computationally bound the machine.
Today, we are bounding the ``omnipresence factor'' of the machine.


\subsubsubsection{Suppose $n = 3, t = 2$ (that is, any two should not get the secret, out of 3 machines)}
This is equivalent to 1 out of 2 (what we did last class), by clubbing two machines together.
So, for this, we will require the existence of trapdoor one-way functions (since this reduces to 1-of-2).


