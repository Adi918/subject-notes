\documentclass[11pt]{article}
%\documentclass[10pt]{llncs}
%\usepackage{llncsdoc}
\usepackage{amsmath}
\usepackage{graphicx}
\usepackage{makeidx}
\usepackage{algpseudocode}
\usepackage{algorithm}
\evensidemargin=0.20in
\oddsidemargin=0.20in
\topmargin=0.2in
%\headheight=0.0in
%\headsep=0.0in
%\setlength{\parskip}{0mm}     
%\setlength{\parindent}{4mm}
\setlength{\textwidth}{6.4in}
\setlength{\textheight}{8.5in}
%\leftmargin -2in
%\setlength{\rightmargin}{-2in}
%\usepackage{epsf}
%\usepackage{url}
\usepackage{epsfig}
\usepackage{tabularx}
\usepackage{latexsym}
\newtheorem{lemma}{Lemma}
\newtheorem{observation}{Observation}
\newtheorem{proof}{Proof}
\newcommand\ddfrac[2]{\frac{\displaystyle #1}{\displaystyle #2}}

\def\qed{$\Box$}
\def\proof{\textit{Proof. }}
\newtheorem{corollary}{Corollary}
\newtheorem{theorem}{Theorem}

\begin{document}
\section{Bit commitment and zero knowledge proofs}

Today's impossibility: blinding and yet binding: we need a binding decision to be made, but the
value that is bound is hidden.


\section{Bit commitment}

Commit phase: 
* Message is hidden.
* Any change to the message is detectable.

Reveal phase:
* message is publicly revealed.
* If we reveal something that is not what we committed, it is detectable.

\section{Zero Knowledge proof}
Assume we can solve bit-commitment.

``I want to prove I know something, but I don't want to reveal what I know''.

Example, Rand Co. We sell random numbers. Our client goes to court, saying that - how do I know that this
company is actually selling random numbers? The client's ulterior motive is to expose the program used by Rand Co.

Court would say: Can you prove that you perform random number generation, without revealing the source code?


\subsection{Zero knowledge with bit commitment}
We can prove that we differentiate taste from 1L bottle from that of 2L bottle, without revealing how.

Proof: I enter a locked room. I cannot see outside. I'll ask you to fill a glass of bottle with either 1L
or 2L. I tell what the water is by drinking it. If I get it correct $n$ times, the odds of that happening was $0.5^n$,
which is astronomically low to happen by luck.


\subsection{Zero knowledge proof for graph 3-coloring}

We now provide a zero knowledge proof system for graph 3-coloring.
since this is NP-complete, we have a ZKP protocol for all NP.

There is a graph $G$. Prover $P$ tells us that there is a 3-coloring of this graph. We, the
verifier $V$ wishes to be convinced of this, but $P$ does not wish to reveal the coloring.

We use the following tools:

1. Random permutation of vertices (Graph $H$ that is isomorphic to $G$).
2. $P$ \textit{commits} the colors of all vertices on the permutation graph, $H$.
3. $V$ asks to reveal the colour of end end points of any one random edge.
With probability of \frac{1}{|E|}, $P$ gets caught.
4. $V$ checks if the edge is properly colored (end vertices have different colors).
5. If this works, then repeat.

Lemma: IP = PSPACE =? ZKP.




\end{document}
