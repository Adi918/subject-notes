\documentclass[11pt]{article}
%\documentclass[10pt]{llncs}
%\usepackage{llncsdoc}
\usepackage{amsmath}
\usepackage{graphicx}
\usepackage{makeidx}
\usepackage{algpseudocode}
\usepackage{algorithm}
\evensidemargin=0.20in
\oddsidemargin=0.20in
\topmargin=0.2in
%\headheight=0.0in
%\headsep=0.0in
%\setlength{\parskip}{0mm}     
%\setlength{\parindent}{4mm}
\setlength{\textwidth}{6.4in}
\setlength{\textheight}{8.5in}
%\leftmargin -2in
%\setlength{\rightmargin}{-2in}
%\usepackage{epsf}
%\usepackage{url}
\usepackage{epsfig}
\usepackage{tabularx}
\usepackage{latexsym}
\newtheorem{lemma}{Lemma}
\newtheorem{observation}{Observation}
\newtheorem{proof}{Proof}
\newcommand\ddfrac[2]{\frac{\displaystyle #1}{\displaystyle #2}}

\def\qed{$\Box$}
\def\proof{\textit{Proof. }}
\newtheorem{corollary}{Corollary}
\newtheorem{theorem}{Theorem}

\begin{document}
\section{Mid 2 topics}

\subsection{1. Hashing schemes}
\begin{Itemize}
\item Hashing schemes
\item Collision Resistance
\item Birthday attacks
\item Merkle Damgard Transform
\item DLP based design of collision resistant hash functions
\end{itemize}

\subsubsection{Questions that are possible}
What is wrong on appending / prepending the length to merkle-damgard?

\subsection{2. Public key crypto}
\begin{Itemize}
\item Key establishmend using diffie-hellman
\item El-gamal
\item RSA
\item Attacks on ``textbook RSA''
\item Padded RSA (a version of which is the current standard)
\end{itemize}


\subsection{Questions possible}
Show that El-gamal is CPA secure.
Attacks on textbook RSA - chinese remainder theorem based attack.


\subsection{3. Oblivious Transfer}
\begin{itemize}
\item From any 1-way permutation.
\item Applied it to perform general secure two-party computation.
\item constructed secure XOR and secure AND.
\end{itemize}


\subsection{Questions possible}
Solve Yao's millionaire problem.

\subsection{4. Secret Sharing}
\begin{itemize}
\item Shamir's algorithm (points on a polynomial of degree $t$)
\item General multiparty secure computation (Genero Rabin Rabin - 1998) (BGW (something gold waterson) - 1988).
\end{itemize}

\subsection{Questions possible}
\begin{itemize}
\item Given shares, reconstruct secret
\item  In the shared domain, construct sharing schemes like what we have constructed.
\end{itemize}

\subsection{5. General access structures}
\item Secret sharing schemes for monotone access structures
\item Computational secret sharing
\end{itemize}


\subsection{Questions possible}
\begin{itemize}
\item Construct weighted shares in the access structure -
      Some people can get more points than others. This is kind of like
      solving for an access structure. Not all access structures can be construced using a
      weighted shamir scheme (proof?)
\item Computationally secret sharing scheme: computationally bounded adversary?
\item 
\end{itemize}

\subsection{6. Secure communication over noisy channel}
\begin{itemize}
\item General 2-party communication without invoking 1way functions.
\item noise based oblivious transfer (since we have 2-party communication).
\end{itemize}

\begin{itemize}
  \item Once we have OT over the channel, solve stuff over the noise.
  \item ``Byzantine agreement over noisy channel'' - If this is possible, then publish! (One-out-of-3 agreement).
        We know that this is possible, from the fact that we can create OT from noise.
\end{itemize}

\subsection{7. Byzantine agreement}
\begin{itemize}
\item Impossibility of 1-out-of-3 case
\end{itemize}


\section{Digital Signatures}

We have $(M, \sigma_k(m))$. $\sigma_k(m)$ is the digital signature.

Appending our signature is *not* going to stop people from copying the signature. So, we need our signature to be
message dependent. That way, a message $M$ relates to the signature of the message.

The world currently uses tree based signatures, Kannan does not find this elegant. So, we will construct RSA
based signatures since they are elegant.

\subsection{Digital signatures are impossible}
We show that byzantine agreement in 1-out-of-3 is impossible. Given digital signatures, this would have been possible.
Hence, digitial signatures are not actually possible.



\subsection{Defining digital signatures}
Digital signature scheme = $<Gen, Sign, Vrfy>$
$Gen(security param) -> <sk, pk>$ (sk = secret key, pk = public key)
$Sign(M, sk) -> \sigma$
$Verify(pk, M, \sigma) -> {yes, no}$

Correctness: if message is signed by the secret key, verify should say yes.
Security: Adversary has access to public key and the security parameter, and the signing algorithm.
          Let Q be the set of all queries that the adversary makes to the signing oracle.
Adversary is supposed to output $<M, \sigma>$ such that the probability such that
$Verify(M, \sigma) = Yes$ is negligible if m \notin Q.

That is, without directly asking for the message's signature, it is not possible to forge the signature.



\subsection{Computationally secure Digital signatures - RSA based}

$RSA_{Sign} = m^d (mod N)$
$RSA_{Verify}(m, \sigma), \sigma ^e mod N =? m$

Encrypt using secret key, decrypt using public key.
This does not satisfy the actual definition, however, since there are some interesting attacks.


\subsubsection{Forging signatures}
Adversary chooses some random $\rho$. Computes $\rho^e (mod N)$.
Now, set $m = \rho^e (mod N), \sigma = \rho$. This passes RSA.

We got a (message, signature) that passes verification. This is a valid attack, but is sort of pointless,
since we have no control over the message.

\subsubsection{Attack on products of messages}

IF we know that $m1, \sigma_1$, $m_2, \sigma_2$ are valid, then
$m_1 m_2, \sigma_1 \sigma_1$ both pass.

We know that:

\begin{align}
m_1^d (mod N) &= \sigma_1 \\
m_2^d (mod N) &= \sigma_2 \\
\text{Hence, } (m_1 m_2)^d (mod N) &= \sigma_1 \sigma_2
\end{align}

This also breaks our security model, since we can write $m = m_1 m_2$, and then construct $\sigma$
from $\sigma_1 sigma_2$.

\subsection{Securing textbook RSA - Hash-and-sign paradigm}

If we have a determinisitic hashing algorithm that has homomorphism properties,
then we simply hash the data and then sign it.

There is a generic proof in the textbook.

\subsubsection{Hashed RSA}

$\sigma = Hash(m)^d (mod N)$
$Verify(m, \sigma) Hash(m) =? \sigma^e (mod N)$

The attacks we had before would not work, since the hash is effectively create a ``barrier'' for the homomorphism.

\subsubsubsection{Attack on products of messages - fails on Hashed RSA}

IF we know that $m1, \sigma_1$, $m_2, \sigma_2$ are valid, then
$m_1 m_2, \sigma_1 \sigma_1$ both pass.

We know that:

\begin{align}
H(m_1)^d (mod N) &= \sigma_1 \\
H(m_2)^d (mod N) &= \sigma_2 \\
\text{However notice that, } (H(m_1) H(m_2))^d (mod N) &= \sigma_1 \sigma_2
\end{align}

We cannot rewrite this as H(m_1 m_2), since H is not homomorphic.

We have a proof that says that if H is indistinguishable from random, then hashed RSA works.

\section{Certificates}
The \textix{verification is public}. Therefore, it becomes ``publically transferable''.

If we trust somebody, and we want to transfer his trust to someone else, how do we do this?


I have some data with me, there is someone I trust, CA (certificate authority).


There is a random person, local browser, who is talking to some bank Z.

how do we know that we were routed to Z's web page?

Z publishes their public key. Adversary(A) also claims they use Z's public key.

How do we know whose public key is correct?

The digital certificate of Z consists of: [Z's public is key is <something>, CA signature].


The adversary will have a certificate of the form: [Z's public key is <something>, CA signature']

So, the idea is that the adversary cannot have a signed digitial certificate.

\subsection{PK infra}
a public key infra is just a collection of CAs and their cooresponding PKs. That is, we need a group of people
to trust initially, who will allow us to trust others.


\end{document}
